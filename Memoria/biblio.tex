\begin{thebibliography}{99}

    \bibitem{bib:navegadores}
    \textit{Navegadores web más empleados:}

    Mozilla: \url{https://www.mozilla.org/}
    
    Opera: \url{https://www.opera.com/}

    Google Chrome: \url{https://www.google.com/intl/es/chrome/}
    
   \bibitem{bib:ros}
    Willow garage y Stanford Artificial Intelligence Laboratory.
    \textit{Página oficial de ROS.}
    \url{https://www.ros.org/}
    
    \bibitem{bib:orca}
    Sourceforge.
    \textit{Página oficial de ORCA:}
    \url{http://orca-robotics.sourceforge.net/}
    
    \bibitem{bib:orocos}
    Orocos.
    \textit{Página oficial de OROCOS:}
    \url{https://orocos.org/}
    
    \bibitem{bib:gazebo}
    Gazebo.
    \textit{Página oficial de Gazebo:}
    \url{https:http://gazebosim.org/}
    
    \bibitem{bib:secundaria}
    Comunidad de Madrid. %Quien lo escribe
    \textit{Asignatura de robótica en Secundaria.} %Nombre
    SIMO EDUCACIÓN, Octubre 2015. % Periodico y fecha de publicacion
    \url{https://n9.cl/7lqc}. % URL

    \bibitem{bib:scratch}
    MIT Media Lab.
    \textit{Página oficial de Scratch:}
    \url{https://scratch.mit.edu/}
    
    \bibitem{bib:lego}
    LEGO.
    \textit{Página oficial de LEGO programming:}
    \url{https://www.lego.com/es-es/categories/coding-for-kids}
   
    \bibitem{bib:gltf}
    \textit{Documentación sobre glTF:}
    \url{https://github.com/KhronosGroup/glTF}
    
    \bibitem{bib:kibotics}
    Kibotics.
    \textit{Plataforma de Kibotics}
    \url{https:https://kibotics.org/}
    
    \bibitem{bib:mindstorm}
    \textit{Especificaciones Lego Ev3}
    \url{https://www.lego.com/es-es/product/lego-mindstorms-ev3-31313}
    
    \bibitem{bib:flask}
    \textit{Documentación de Flask}
    \url{https://flask.palletsprojects.com/en/1.1.x/}
    
	\bibitem{bib:Django}
    \textit{Documentación de Django}
    \url{https://www.djangoproject.com/es}

    \bibitem{bib:aframe}
    Diego Marcos, Don McCurdy, Kevin Ngo. Supermedium, Google y WebVR.
    \textit{Documentación A-Frame.}
    \url{https://aframe.io/}. 

    \bibitem{bib:blockly}
    Google.
    \textit{Documentación Blockly.}
    \url{https://developers.google.com/blockly}
    
    \bibitem{bib:wwwschools}
    Refsnes Data.
    \textit{Documentación para el desarrollo web.}
    \url{https://www.w3schools.com/}
    

\end{thebibliography}