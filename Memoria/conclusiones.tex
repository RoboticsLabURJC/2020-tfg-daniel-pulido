\chapter{Conclusiones}
\label{chap:conclusiones}

En este capítulo se recogen las conclusiones a las que se ha llegado una vez realizado el Trabajo de Fin de Grado, y voy a valorar los logros alcanzados, comparándolos con los objetivos planteados. Así como una recapitulación sobre los conocimientos adquiridos y se presentarán una serie de mejoras para realizar en un futuro.

\section{Conclusiones}
\label{sec:conclusiones}
El objetivo principal era  integrar el robot \textit{LEGO EV3} en la plataforma de \textit{Kibotics} el cual ha sido alcanzó con éxito este objetivo lo dividimos en tres subobjetivos que vamos a ir analizando.

El primer subobjetivo consistía en dar soporte a la parte simulada del robot  \textit{LEGO EV3} en \textit{WebSim}. En el Capitulo 4 se explica cómo se ha llevado a cabo, aportando los modelos en 3D, los \textit{drivers} y bloques para las nuevas funcionalidades que añade el \textit{LEGO EV3}. Entre ellos está el nuevo sensor de contacto con su función \textit{IsTouching} y actualización de la funciones en los \textit{drivers} de los sensores de \textit{Ultrasonidos} y \textit{Sensor de IR}.\\

El segundo subobjetivo era dar soporte al robot real para aceptar código desde el navegador web y que el \textit{LEGO EV3} fuera capaz de recibir ese código, interpretarlo y ejecutarlo en local. Para ello se crearon unos \textit{drivers} en Python que daban la funcionalidad al robot real. Para que el robot real pudiera ejecutar programas que le enviaban, necesitó que se le instalara una imagen de ditribución \textit{Linux}, se instaló los \textit{drivers} nativos al \textit{LEGO} y se lanzó un servidor \textit{Flask} que se encargaba de recibir los paquetes \textit{HTTP} que enviaba el navegador Web con el código en \textit{Python}. Una vez recibido se encargaba de guardarlo como ejecutable y lanzarlo en local. Todo esto viene explicado con más detalle en el Capítulo 5\\

El tercer subobjetivo era incluir ejercicios que actuaran como validación experimental de los avances que he ido explicando anteriormente, y además para tener una batería de ejercicios como temario. Se han implementado: \textit{Bump and Go, AtraviesaBosque, SigueLineasIR } para el robot simulado, además del \textit{Bump and Go y SigueLíneas} para el robot real en \textit{Python}

\section{Mejoras futuras}
\label{sec:mejoras_futuras}

La implementación del soporte para el \textit{LEGO EV3} en \textit{Kibotics} ha supuesto un avance para la plataforma, pero además ha dejado las puertas a futuros proyectos que pueden enriquecer más la plataforma, como por ejemplo:

\begin{itemize}
    \item Añadir el \textit{LEGO EV3} real al catálogo de robots que ofrece \textit{Kibotics} en \textit{producción}, y que se pueda programar y lanzar desde la propia página web de \textit{Kibotics} en vez de desde una página externa  
    \item Añadir más ejercicios a la plataforma con los sensores que ofrece \textit{LEGO}, aparte de los que vienen con el paquete \textit{MINDSTORM} como pueden ser el micrófono, el sensor de temperatura, o el sensor de luz ambiental. 
    \item Añadir más montajes con el \textit{LEGO Ev3} de modo que el propio robot simulado tenga un centro de gravedad y pueda caer para que el \textit{girosensor} tenga más utilidad. O que haya un nivel de luz, o temperatura en el escenario, lo que daría mas posibilidades a la hora de crear ejercicios. Añadir más diseños de montajes 3D de robots de \textit{LEGO EV3} y con ellos crear más tipos diferentes de ejercicios.
 
 \end{itemize} 
