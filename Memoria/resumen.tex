\chapter*{Resumen}
\markboth{RESUMEN}{RESUMEN} 

     La robótica proporciona cada vez aplicaciones más útiles para la sociedad, como coches autónomos, aspiradoras robóticas, etc... Además de la necesidad de profesionales formados en este sector, la robótica educativa se ha mostrado estos últimos años como una estupenda herramienta para introducir a los niños en la tecnología de una manera atractiva. Además varios estudios han demostrado que la robótica educativa mejora las habilidades en física, matemáticas y tecnología. 
\\  
  
El proyecto se ha centrado en la integración del robot \textit{LEGO Ev3} en plataforma de robótica educativa \textit{Kibotics}, que también soporta varios robots reales como \textit{mbot} o el drone \textit{Tello}. \textit{Kibotics} utiliza un simulador llamado WebSim, que esta basado en el entorno de \textit{A-Frame}, para representar los ejercicios que contiene la plataforma. Para la integración de este nuevo robot se han creado tres modelos tridimensionales con tres diferentes sensores en el entorno de \textit{WebSim}. Se han creado los \textit{drivers} en \textit{JavaScript} necesarios para que el robot simulado sea programable dentro de la plataforma. También se han creado \textit{drivers} en \textit{Python} para el robot real. Todo lo implementado ha sido validado experimentalmente con la creación de varios ejercicios educativos. En ellos las aplicaciones de los usuarios acceden a los sensores y actuadores del robot \textit{LEGO Ev3}, tanto el simulado como el real, usando esos \textit{drivers}.  \\
    
    
    Además de los \textit{drivers} se ha integrado el soporte para el \textit{LEGO} real haciendo que el código fuente que el usuario escribe en la página web de la plataforma, junto con el propio \textit{driver} en \textit{Python}, se envíen al robot a través de peticiones HTTP y se descarguen automáticamente en un servidor Flask que corre a bordo del \textit{LEGO} y se ejecuten allí.  