\chapter*{Resumen}
\markboth{RESUMEN}{RESUMEN} 

     Este Trabajo de Fin de Grado tiene como fin la mejora de la plataforma de robótica educativa \textit{Kibotics}. Plataforma que esta destinada a la educación de la robótica para alumnos desde niños hasta adolescentes, ya que cada vez es más común el uso de nuevas tecnologías en las etapas más tempranas del aprendizaje, porque está demostrado que mejora las habilidades en física, matemáticas y tecnología. \textit{Kibotics} utiliza un simulador llamado WebSim, que esta basado en el entorno de \textit{A-Frame}, para representar los ejercicios que contiene la plataforma\\
    
    
    El proyecto se ha centrado en la integración del robot educativo \textit{LEGO Ev3} en plataforma de \textit{Kibotics}. Para ello, se han creado tres modelos tridimensionales con tres diferentes sensores para implementarlos en el entorno de \textit{WebSim}. Se han creado los \textit{drivers} en \textit{JavaScript} y funciones al \textit{RobotApi} necesarios para que el robot sea programable dentro de la plataforma. También he creado \textit{Drivers} en robot real para que sea capaz de ejecutar código en \textit{Python} que le llega al robot mediante peticiones de HTML. Todo lo implementado ha sido validado con la creación de ejercicios.  \\
    
    
    Las implementaciones en el \textit{drivers} simulado se han realizado en \textit{JavaScript}, tanto el servidor \textit{Flask} como los \textit{drivers} de robot real se han programado en \textit{Python}.  