\chapter{Introducción}
\label{chap:intro}
En este capítulo se introducen los conceptos básicos en robótica, de como esta nos ayuda en nuestro día a día, y cual es su estado actual. Y como puede ser un gran recurso en la educación. En lo que se basa este proyecto

\section{Robótica}
\label{sec:robotica}
La robótica es una rama de las ingenierías y de las ciencias de la computación que se encarga del diseño, construcción, operación, estructura, manufactura y aplicación de los robots.
El término \textit{robot} se popularizó con el éxito de la obra R.U.R. (\textit{Robots Universales Rossum}), escrita por Karel Čapek en 1920. En la traducción al inglés de dicha obra la palabra checa \textit{robota}, que significa trabajos forzados o trabajador, fue traducida al inglés como robot.
Un robot es una entidad virtual o mecánica artificial. Están diseñados con un proposito propio.La independencia creada en sus movimientos hace que sus acciones sean la razón de un estudio razonable y profundo en el área de la ciencia y tecnología. La palabra robot puede referirse tanto a mecanismos físicos como a sistemas virtuales de software, aunque suele aludirse a los segundos con el término de bots.

    \begin{figure}[H]
    \centering
    \includegraphics[width=0.8\textwidth]{img/robot.png}
    \caption{Imagen clásica de un robot} \label{fig:robot}
    \end{figure}

No hay un consenso sobre qué máquinas pueden ser consideradas robots,
dentro de este proyecto tomaremos como definición que un robot es un sistema autónomo programable capaz de realizar tareas complejas. Además, todos los robots se componen de tres partes esenciales  se componen de sensores, controladores y actuadores.
\begin{itemize}
    \item \textbf{Sensores}: Son los sentidos del robot, con ellos ve, escucha y sabe lo que hay en el entorno. Recogen la información necesaria para que el robot realice la tarea En este grupo se encuentran lásers, cámaras, ultrasonidos u odómetros..
    \item \textbf{Controladores}: El equivalente al cerebro humano, utiliza los datos recogidos por los sensores para elaborar una respuesta para que la lleve acabo los actuadores.
    \item \textbf{Actuadores}: Equivalen a los músculos humanos, son los que se encargan de interactuar con el entorno para llevar a cabo su tarea. Son brazos mecánicos, motores, etcétera...
\end{itemize}

\subsection{Aplicaciones robóticas}

Ahora que tenemos las bases de lo que es un robot asentadas podemos hablar de cuales son los principales propósitos de los robots hoy en día,aunque la mayor parte de ellos son utilizados por empresas en labores industriales. Aunque hay otros que podemos encontrar en nuestra vida cotidiana, en casas, hospitales, almacenes de tiendas... Esto es debido a la precision de algunos trabajos, la eficiencia en el trabajo, la reducción de costes que supone o que pueden realizar acciones de alto riesgo para las personas. Los ejemplos mas famosos en estos campos son los siguientes:
\begin{itemize}
    \item Robots Domésticos: Creados para realizar las tareas del hogar. Los mas famosos y destacados en el mercado son los Robots \textit{Roomba}, aspiradores autónomos, y también el primer robot que se ha comercializado para todos los públicos y de manera global. Un gran paso para la robótica

\begin{figure}[H]
    \centering
    \includegraphics[width=0.6\textwidth]{img/roomba.jpg}
    \caption{Aspiradora robótica \textit{Roomba}} \label{fig:roomba}
    \end{figure}

    \item Robots médicos: Son robots diseñados para el uso en medicina para realizar tareas que requieren mucha precision como en el caso de una cirugía, con el robot \textit{Da Vinci} o robots diminutos que son capaces de navegar por las venas hasta llegar al corazón y allí realizar la cirugía necesaria.
      \begin{figure}[H]
    \centering
    \includegraphics[width=0.6\textwidth]{img/davinci.jpg}
    \caption{Robot médico \textit{Da Vinci}} \label{fig:davinci}
    \end{figure}
    
    \item Robots militares: Son robots orientados a tareas militares, como reconocimientos de zonas conflictivas o rescate de personas, desactivación de bombas. En los últimos años también se han desarrollado mucho los drones en combate.
      \begin{figure}[H]
    \centering
    \includegraphics[width=0.6\textwidth]{img/bigdog.jpg}
    \caption{Robot militar \textit{Big Dog} creado por \textit{Boston Dynamics}} \label{fig:bigdog}
    \end{figure}
    
    \item Vehículos autónomos: Es el campo de la robótica que más en auge esta ahora mismo. El objetivo de estos robots es usar la información que proporcionan sus sensores internos, como cámaras, sensores infrarrojos \textit{Lidar}, y sensores externos como el GPS para llevar de un punto a otro un vehículo.

    \begin{figure}[H]
        \centering
        \includegraphics[width=0.7\textwidth]{img/waymo.jpg}
        \caption{Vehículo \textit{Waymo} de \textit{Google}} \label{fig:waymo}
    \end{figure}

    \item Robots Espaciales: Los famosos \textit{Rover} de la \textit{NASA} son robots diseñados para entornos donde el ser humano no puede llegar. Se centran en reconocimiento del terreno y análisis de las muestras que recogen.
        \begin{figure}[H]
    \centering
    \includegraphics[width=0.5\textwidth]{img/perseverance.jpg}
    \caption{Robot \textit{Perseverance} de la \textit{NASA}} \label{fig:Perseverance}
    \end{figure}
\end{itemize}

\subsection{Software en robótica}
\label{subsec:softwarerobot}
Para dotar de esta inteligencia a los robots se necesitan herramientas que transformen los datos recibidos de los sensores en algo que puedan aplicar en los actuadores. Hace años, cada maquina tenía un software especifico con sensores y actuadores únicos para ese robot y esa tarea a desarrollar. Esto hacia, que aunque hubieras implementado el software para otros robots anteriormente, tuvieras que repetir el proceso con cada nuevo robot. Con los años se desarrollaron plataformas de software que permiten desarrollar de manera genérica para todos los robots, y actuando de mediador entre el robot y el software del creador,estos son los llamados \textit{middleware} que hacen que te puedas abstraer de los \textit{drivers} característicos de cada robot. Los middleware mas importantes a día de hoy son:
\begin{itemize}
    \item \textit{\textbf{Robot Operating System (ROS)}}\cite{bib:ros}. Plataforma de \textit{software} libre para el desarrollo de \textit{software} de robots. Provee servicios estándar de un sistema operativo como la abstracción de \textit{hardware}, control de dispositivos de bajo nivel, mecanismos de intercambio de mensajes entre procesos y mas herramientas vitales para el desarrollo del robot. Es el mas utilizado a día de hoy porque fue especialmente desarrollado para \textit{UNIX} y luego se implemento para el resto de sistemas operativos
    \item \textit{\textbf{ORCA}}\cite{bib:orca}. Plataforma de \textit{software} libre diseñado para crear aplicaciones mas complejas, ya que esta orientado a las componentes por separado
     \item \textit{\textbf{OROCOS}}\cite{bib:orocos} Proyecto de \textit{software} libre también orientado a componentes y basado en C++
      \item \textit{\textbf{JdeRobot}}\cite{bib:jderobot} Plataforma de desarrollo robótico, en la que se basa este proyecto. Tiene varios nodos programados  con varios lenguajes de programación, con compatibilidad con otros \textit{middleware}.
\end{itemize}{}


\section{Robótica educativa}
\label{sec:educativa}
La robotica educativa ha ido tomando mas importancia con los años, ya que cada vez es mas importante que estudiantes de cualquier nivel estén familiarizados con la tecnología, tiene valores positivos como la implementación de pensamiento lógico, resolución de problemas y trabajo en equipo en las actividades académicas, que son ramas del conocimiento que se desarrollan poco en edades tempranas , con una componente en conocimiento matemático y físicos y ademas añade un atractivo que no tienen las asignaturas convencionales.
Muchos estudios han demostrado que el uso de kits de robótica en la educación favorece a la capacidad de reflexión de los estudiantes.
Cada año se crean mas cursos de robótica, y en 2015 la comunidad de Madrid introdujo la asignatura de robótica en los planes docentes de Enseñanza Secundaria con la asignatura ``Tecnología, Programación y Robótica''\cite{bib:secundaria} y en el curso 2020-2021 se empezará a implantar en Educación Primaria la asignatura ``Programación y Robótica''\cite{bib:primaria}.\\
Una de las mayores partes de la robótica tiene que ver con la programación, que ademas de ser una habilidad muy importante para la sociedad actual, es algo complejo. Por lo que se utilizan lenguajes de programación visual, estos se tratan de lenguajes que abstraen en bloques las funciones o métodos de cualquier lenguaje de programación. Dentro de este tipo de lenguajes, los mas destacables son :

\begin{itemize}
    \item \textit{\textbf{Scratch}}\cite{bib:scratch}: proyecto liderado por el Grupo \textit{Lifelong Kindergarten} del \textit{MIT}, es utilizado por estudiantes para programar animaciones, juegos e interacciones. Su atractivo reside en lo facil que es de entender el pensamiento computacional debido a su sencilla interfaz gráfica y la implementación de sus bloques.
    \begin{figure}[H]
    \centering
    \includegraphics[width=0.7\textwidth]{img/scratch.jpg}
    \caption{Interfaz gráfica de Scratch} \label{fig:scratch}
    \end{figure}

    \item \textit{\textbf{LEGO}}\cite{bib:lego}: Es el robot base de este proyecto,dispone de una amplia gama de robots programables y cada uno de ellos tiene un sistema gráfico, que es similar entre ellos pero también ligado a la edad el estudiante para el que esta diseñado el software. 

    \begin{figure}[H]
    \centering
    \includegraphics[width=0.7\textwidth]{img/lego1.png}
    \caption{Interfaz de LEGO WeDo} \label{fig:lego1}
    \end{figure}
    
Por ejemplo en la figura \ref{fig:lego1} se puede ver que la interfaz en este caso, es con colores vivos, los cuales representan distintas funcionalidades dentro del robot, es decir, el amarillo representa las acciones propias de programación, como: inicio de programa, fin de programa, bucles, esperar, etcétera. El color rojo representa los sensores del robot, todo lo que recoja datos. Y el color verde representa los motores que equivalen a los actuadores en este robot.
Como se puede observar es una abstracción muy simple para estudiantes de mas corta edad.
	 \begin{figure}[H]
    \centering
    \includegraphics[width=0.7\textwidth]{img/lego2.jpg}
    \caption{Interfaz gráfica de LEGO Ev3} \label{fig:lego2}
    \end{figure}

En el caso del software para el \textbf{LEGO Ev3}, añade un grado de complejidad, incluyendo apartados para realizar operaciones matemáticas, envio de archivos entre robots, y añade mas actuadores, como la pantalla que integra el robot, o los altavoces.

En el caso de LEGO y en otros kits incorporan los elementos básicos para la construcción de un robot. En este en particular viene con lo indispensable para construir con piezas de LEGO. También incluye un microprocesador para ser programado, con Linux instalado, sensores (infrarrojos, táctiles y de color) y motores.
  \begin{figure}[H]
    \centering
    \includegraphics[width=0.7\textwidth]{img/kitev3.jpg}
    \caption{Kit de piezas y sensores de LEGO Ev3} \label{fig:lego3}
    \end{figure}
\end{itemize}{}


En el siguiente capitulo, profundizare mas en lo que se puede hacer con el robot \textbf{LEGO Ev3} y explicare cuales van a ser los objetivos y porque elegir este robot.


